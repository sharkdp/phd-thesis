\chapter*{Introduction}
\addcontentsline{toc}{chapter}{Introduction}

% Resources
% - http://benasque.org/2014numerical/talks_contr/021_BenasqueBergholtzPart1.pdf
% - Bernevig book (cite)
% http://www.uam.es/personal_pdi/ciencias/alevyyey/FMC/phystoday.pdf

%%% History
The history of topological materials is just a little over thirty years old.
A good point to start is the discovery of the quantized Hall conductance in two-dimensional semiconductor samples by von Klitzing in the early 1980s~\cite{Klitzing1980,Klitzing1992}.
Remarkably, the conductance is given by an integer multiple of a fundamental constant that depends on the elementary charge and Plancks constant and is independent of material properties or external conditions.
Due to the high precision of the quantization levels, for which an explanation was given in the following years by Laughlin and Halperin~\cite{Laughlin1981,Halperin1982}, this effect has found immediate applications in metrology as a direct measurement of the fine structure constant and a standard for the unit of resistance.

Thouless and others subsequently found a surprising connection between the quantized Hall conductance and topological invariants~\cite{Thouless1982,Niu1985,Kohmoto1985,Kohmoto1989,Bellissard1994}.
In particular, an integer quantum Hall sample is an example of a topological insulator, a material that is insulating in much the same way as an ordinary insulator -- but has conducting states at the edge.

A related but considerably more complex phenomenon was discovered by Tsui \emph{et. al.} in 1982 \cite{Tsui1982}.
They found that the Hall conductance in... also developed plateaus at certain fractional values.

% Haldane model "quantum anomalous Hall effect"
\cite{Haldane1988}

% topological quantum computation
\cite{Kitaev2003}

% TRI topological insulators
\cite{Kane2005a,Kane2005,Hasan2010}

% experimental discovery of spin-Hall effect:
Zhang, Bernevig (science)

% 5/2 state: measured in 1987 and still remains unexplained -> need for model systems
\cite{Willett1987}

%%% Realization in ultracold systems

% realizing artificial magnetic fields (experiments)
see dip fermions references + update
Spielman: PRL 114, 125301 (2015)

% reaching the strong regime
\cite{Aidelsburger2011,Aidelsburger2013,Miyake2013}

% realizing topological states:
Bloch: Berry phase experiment
Haldane model \cite{Jotzu2014}

% dipolar spin systems
% - Hazzard2014
% - Barredo2014



% idea behind topological states: analogy with Gauss bonnet
% experimental works on realizing spin systems (non-dipolar):~\cite{Fukuhara2013,Simon2011}
% dipolar:~\cite{DePaz2013}
% new effects in spin systems due to dipolar:~\cite{Avellino2006,Hauke2010,Peter2012b,Hazzard2014}
