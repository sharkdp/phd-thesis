\chapter{Rydberg electron induced BEC losses}
\label{rydberg_losses}

[...]

For a detailed overview on this topic, see \refscite{Balewski2013,Balewski2014} as well as~\cite{Karpiuk2014,Gaj2014}.






\section{Interaction between electron and ground state atoms}

In the s-wave approximation, the contact interaction between the electronic density $\rho(\vecr)=\absvsq{\Psi(\vecr)}$ in the Rydberg $n$s state and the ground state atoms is described by the interaction potential $V(\vecr)=g \rho(\vecr)$, where $g=2\pi \hbar^2 a/\mu$ is the coupling constant that is determined by the electron-atom scattering length $a$ and the reduced mass $\mu \approx m_e$. Within a local density approximation with a constant atomic density
\begin{align}
    n(\vecr)=\frac{1}{V} \sum_{\vecp,\vecq} \ef{i\vecq\vecr}\aopd_{\vecp+\vecq} \aop_{\vecp}
\end{align}
with annihilation operators $\aop_{\vecp}$ and quantization volume V, the interaction can be expressed as a convolution in momentum space, where
\begin{align}
H_\text{int} = g\intvol n(\vecr)\rho(\vecr) = \frac{g}{V} \sum_{\vecp,\vecq} \aopd_{\vecp+\vecq} \aop_{\vecp} \rhoq.
\end{align}
Neglecting constant energy shifts and two-particle excitations, we can write the interaction in terms of Bogoliubov operators $\bop_{\vecq}=\bogu_q \aop_{-\vecq} - \bogv_q \aopd_{\vecq}$ and the BEC particle number $N_0$ as
\begin{align}
H_\text{int}\approx\frac{g\sqrt{N_0}}{V} \sum_{\vecq\ne 0} \rho^{\phantom\dagger}_{\vecq} \big(\bogu_{q}-\bogv_{q}\big) \big(\bopd_{\vecq}+\bop_{-\vecq}\big).
\end{align}
\section{Atom losses}
To estimate the number of excitations induced by the presence of the Rydberg electron, which has a finite lifetime of $\tau=1/\gamma$, we first consider the probability to excite a certain mode with quasi momentum~$\vecq$, when a perturbation of the type $H_\text{int} \ef{-\gamma t}$ is applied. In lowest order we have
\begin{align}
P_{0\rightarrow\vecq}&=\Bigg|-\frac{i}{\hbar}\integralb{0}{\infty}{t}\ef{i\omega_{q}t-\gamma t}\braketop{\vecq}{H_\text{int}}{0}\Bigg|^2.
\end{align}
Here, $\ket{0}$ describes the many particle ground state and $\ket{\vecq}=\bopd_{\vecq}\ket{0}$ is the excited state with energy
\begin{align}
    E_{q}=\hbar\omega_{q}=\sqrt{\epsilon_{q}^2+2n_0g_c\epsilon_{q}}.
\end{align}
Here, we have introduced the recoil energy $\epsilon_{q}=\hbar^2 q^2/2m_\text{Rb}$, the BEC density $n_0=N_0/V$ and the atom-atom coupling constant $g_c=4\pi \hbar ^2a_{\text{Rb}}/m_{\text{Rb}}$ with the s-wave scattering length $a_{\text{Rb}}$.
For the probability we find
\begin{align}
P_{0\rightarrow\vecq}=\frac{g^2 \rho_{\vecq}^2 }{V^2\hbar^2}   \integral{\omega}S(\vecq,\omega)\absvsq{C(\omega)} = \frac{g^2 \rho_{\vecq}^2}{V^2\hbar^2} N_0 \frac{\epsilon_{q}}{E_{q}} \absvsq{C(\omega_q)},
\end{align}
where $S({\vecq},\omega)=N_0 \,\epsilon_{q}/E_{q} \cdot \delta(\omega-\omega_{q})$ is the dynamic structure factor of the BEC and $C(\omega)=1/(\gamma-i \omega)$ is the Fourier transform of the exponential decay.
During the time-of-flight process, the atom-atom interactions quickly become negligible and the Bogoliubov modes are converted in to real particles. Using $N=\sum_{\veck} \aopd_{\veck}\aop_{\veck}$, we find that $\braketop{\vecq}{N}{\vecq}-\braketop{0}{N}{0} = u_{q}^2+v_{q}^2$ additional particles are in the excited state. The total number of lost atoms therefore be expressed as
\begin{align}
L &= \sum_{\vecq} P_{0\rightarrow\vecq}\, (u_{q}^2+v_{q}^2).
\end{align}
Replacing the sum by an integral, we find
\begin{align} \tlabel{losseseq}
L&= \frac{1}{2\pi^2}\frac{n_0 g^2}{\hbar^2} \integral{q} q^2\rho_q^2\,\frac{1+(q\xi)^2}{2+(q\xi)^2} \, \absvsq{C(\omega_q)}  \equiv \integral{q} P(q) (u_{q}^2+v_{q}^2)
\end{align}
where $\xi=1/\sqrt{8\pi n_0 a_{\text{Rb}}}$ is the healing length of the condensate. For high principal quantum numbers we can use an asymptotic expression for the Fourier transform of the electronic density  (see \tref{fourier} for details)
\begin{align}
    \rho_q=J_0(q R_e/2) \operatorname{sinc}(q R_e/2)
\end{align}
where $R_e=2n^2 a_0$ denotes the classical electron radius. Then, we find
\begin{align} \tlabel{lostpertau2}
L/\tau^2 = \frac{2}{\pi^2}\frac{n_0 g^2}{R_e^2 \hbar^2}\integral{q} J_0(q R_e/2)^2 \sin(q R_e/2)^2 \,\frac{1+(q\xi)^2}{2+(q\xi)^2} \frac{1}{1+\omega_q^2/\gamma^2},
\end{align}
where we have separated the main dependency on the two experimentally accessible quantities on the left hand side.
To understand what kind of excitations are generated by the Rydberg electron, \tref{modes} shows the excitation weight $P(q)\sim P_{0\rightarrow q} q^2$ as a function of $q$. The main excitation peak is located at $q\approx 2/R_e < 1/\xi$, which lies well in the phonon regime for all principal quantum numbers investigated in the experiment.

\fig{.8}{modes}{Weight of the different excitation momenta $q$ for two principal quantum numbers $n=110$ and $160$. The Bogoliubov excitation spectrum with linear and quadratic regimes is shown as a reference.}

Some experimental details require extensions to \tref{lostpertau2} given above. They are described in the following.

\paragraph{Atomic density:}
First, to account for density inhomogeneities due to the external potential in a simple way, we replace the BEC density $n_0$ by its mean value
\begin{align}
    \overline{n} = n_0 \bc{1-\bb{ \frac{2R_e}{5R_\rho} }^2 - \bb{ \frac{R_e}{5R_z} }^2}
\end{align}
on a sphere of radius $R_e$ centered in the middle of the cylindrical cloud with Thomas-Fermi radii $R_\rho$ and $R_z$ in radial and axial direction, respectively.

\paragraph{Field ionization:}
Second, in the experimental sequence, the interaction between the Rydberg electron and the ground state atoms is suddenly terminated after a certain time $t_c$ at which the field ionization occurs. To account for this, the function $C(\omega)$ is modified accordingly:
\begin{align}
\absvsq{C(\omega)}=\Bigg| \integralb{0}{t_c}{t}\ef{i\omega t-\gamma t} \Bigg|^2=\frac{1+\ef{-2\gamma t_c} -2 \ef{-\gamma t_c} \cos(\omega t_c)}{\gamma^2+\omega^2}.
\end{align}

\paragraph{Lower cutoff:}
The last correction concerns the way the losses are detected in the experiment. In the absorption images, excitations at small momenta cannot be distinguished from the condensate fraction due to finite momentum components in the Thomas-Fermi profile. A lower cutoff may thus be introduced in the radial $q$ integration. It turns out that this correction is negligible and almost all excitations will be detected as losses.

Finally, \tref{losses} shows the quantity $L/\tau^2$ in a comparison between experiment and theory. The agreement is reasonably well if the measured lifetimes are taken into account (see also~\cite{Karpiuk2014}).

\fig{.7}{losses}{Atom losses divided by the square of the Rydberg lifetime $\tau$. Comparison between experiment (see~\cite{Balewski2014}) and theory. For the solid line, a lifetime of $\tau=10\si{\micro\second}$ is assumed while the crosses take the measured Rydberg lifetime into account.}

\section{Fourier transform of the electronic density}
To find the number of losses in \tref{losseseq}, the Fourier transform of the electronic density in the Rydberg state is required. To calculate this quantity, we start with the wave function of the Hydrogen $n$s state which is given by
\begin{align}
\Psi(r)=\frac{\ef{-\frac{r}{n}}}{\sqrt{\pi\, n^5}} \, L_{n-1}^1\!\bb{\frac{2 r}{n}}.
\end{align}
In this section, we have set $a_0=1$ to avoid cluttering of notation.
The three-dimensional Fourier transform for spherically symmetric functions is directly given by a Hankel transformation in the radial coordinate. With $\rho(r)=|\Psi(r)|^2$, we have
\begin{align}
    \rho_q&=\intvol \ef{-i \vecq \vecr} \rho(r) =\frac{4\pi}{q}\integralb{0}{\infty}{r} r \sin(qr) \rho(r) \\
    &=\frac{4}{q\,n^5}\integralb{0}{\infty}{r} r \sin(qr) \ef{-\frac{2r}{n}} \, L_{n-1}^1\!\bb{\frac{2 r}{n}}^2 \\
    &=\frac{1}{q\,n^3}\integralb{0}{\infty}{x} x \sin\!\bb{\frac{qn}{2}x} \ef{-x} \, L_{n-1}^1\!\bb{x}^2
\end{align}
where we have used the reduced length $x=\frac{2r}{n}$ in the last step.
\subsection{Universal solution in the classical limit}
For large $n \goesto \infty$ we expect the Fourier transform $\rho_q$ to be a universal function of the rescaled momentum $k=2n^2 q$, where the factor $2n^2$ is the classical electron radius (in units of $a_0$). Using this transformation, we find
\begin{align}
\rho_k = \frac{2}{k n}\integralb{0}{\infty}{x} x \sin\!\bb{\frac{kx}{4n}} \ef{-x} \, L_{n-1}^1\!\bb{x}^2.
\end{align}
With the explicit expression for the Laguerre polynomial
\begin{align}
L_{n-1}^1\!\bb{x}=\sum_{\alpha=1}^{n} \binom{n}{\alpha} \frac{(-x)^{\alpha-1}}{(\alpha-1)!}
\end{align}
we can expand the square $L^1_{n-1}(x)^2$:
\begin{align}
\rho_k=\frac{2}{k n} \sum_{\alpha=1}^{n}\sum_{\beta=1}^{n} \frac{(-1)^{\alpha+\beta}}{(\alpha-1)! (\beta-1)!} \binom{n}{\alpha}\binom{n}{\beta} \integralb{0}{\infty}{x} \sin\!\bb{\frac{kx}{4n}} \ef{-x} x^{\alpha+\beta-1}.
\end{align}
We can now evaluate the remaining integral
\begin{align}
    \integralb{0}{\infty}{x} \sin\bb{\frac{kx}{4n}} \ef{-x} x^{\alpha+\beta-1} &= \Im\!\bc{ \integralb{0}{\infty}{x} \ef{-(1-ik /4n)x} x^{\alpha+\beta-1}} \\
    &= \Im\!\bc{ \frac{(\alpha+\beta-1)!}{\bb{1-\frac{i k}{4n}}^{\alpha+\beta}} }
\end{align}
leading to
\begin{align}
    \rho_k=\frac{1}{2n^2} \sum_{\alpha=1}^{n}\sum_{\beta=1}^{n} (-1)^{\alpha+\beta} \frac{(\alpha+\beta-1)!}{(\alpha-1)! (\beta-1)!} \binom{n}{\alpha}\binom{n}{\beta} \Imf{ \frac{1}{\kappa\bb{ 1-i\kappa }^{\alpha+\beta}} }
\end{align}
where we have defined $\kappa=\frac{k}{4n}$ to simplify the structure.
We can now expand this expression into a series around $\kappa=0$.
All odd orders vanish identically.
For even $\nu$, the coefficient of $\nu$-th order is given by
\begin{align}
\rho^{(\nu)}_k/\nu! &=\frac{(-1)^{\nu/2}}{2n^2 (4n)^\nu (\nu+1)!} \sum_{\alpha=1}^{n}\sum_{\beta=1}^{n} (-1)^{\alpha+\beta} \frac{(\alpha+\beta+\nu)!}{(\alpha-1)! (\beta-1)!} \binom{n}{\alpha}\binom{n}{\beta}.
\end{align}
Finally, we can take the classical limit $n \goesto \infty$, allowing us to express the double infinite series as
\begin{align}
\rho^{(\nu)}_k/\nu! &= \frac{(-1)^{\nu/2}}{(4n)^\nu (\nu+1)!} \binom{2\nu+1}{\nu}.
\end{align}
The power series in $\nu$ can now be summed to give the final result
\begin{align}
\rho_k &= \sum_{\nu=0,2,\dots}^\infty \frac{(-1)^{\nu/2}}{(4n)^\nu (\nu+1)!} \binom{2\nu+1}{\nu} k^\nu = J_0\!\bb{\frac{k}{2}}\text{sinc}\!\bb{\frac{k}{2}}
\end{align}
where $J_0$ is the zeroth-order Bessel function. Transforming back to the momentum variable $q$, we find a concise form for the Fourier transform of the electronic density in the classical limit:
\begin{align} \tlabel{fourier}
\rho_q = J_0\!\bb{q n^2}\text{sinc}\!\bb{q n^2}.
\end{align}

\subsection{Classical probability distribution}
\fig{0.7}{classicalprob}{Radial probability distribution $P_n(r)$ for principal quantum number $n=10, 20$ and $40$. The black solid line shows the classical probability distribution $P_\infty(r)$ which diverges at the classical turning point $R_e=2n^2a_0$. The envelope is given by $2P_\infty(r)$.}
An interesting application of the expression for $\rho_q$ in \tref{fourier} is to derive the classical probability function of the Hydrogen atom. By Fourier transforming the universal function $\rho_k$ back to real space we find
\begin{align}
\rho(x)=\frac{1}{16\pi^2\, n^6 \, x^{3/2} (1-x)^{1/2}}
\end{align}
as a function of the reduced coordinate $x=r/R_e=r/2n^2$. As expected, the probability distribution diverges at the classical turning point $x=r/R_e=1$. It can easily be checked, that $\rho(x)$ is properly normalized:
\begin{align}
4\pi \integralb{0}{R_e}{r} \rho(r/R_e) r^2 = 1.
\end{align}
The radial probability function is given by
\begin{align}
P(r)=4\pi \rho(r/R_e) r^2 = \frac{2}{\pi R_e} \sqrt{\frac{r/R_e}{1-r/R_e}}
\end{align}
and is shown in \tref{classicalprob}, where it is compared with the exact expressions for a finite principal quantum number $n$.
