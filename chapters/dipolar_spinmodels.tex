\chapter{Introduction: dipolar spin models}

% \section{Dipolar spin systems}
% \subsection{Lattice structures}
% \subsubsection{One-dimensional}
% \subsubsection{Two-dimensional}

% \section{Topological band structures}
% \subsection{Chern number}
% \subsection{Edge states}
% \subsection{Quantum Hall effect}

\section{Physical realizations}
\todo{NV centers, Rydberg atoms, polar molecules, dipolar atoms}


\section{Dipole-dipole interaction}
The aim of this introductory section is to rewrite the familiar interaction between two dipoles $\vec{d}_i$ and $\vec{d}_j$ at positions $\vecR_i, \vecR_j$,
\begin{align} \tlabel{ddint}
\Hdd(\vecR_{ij}) = \frac{\kappa}{\absv{\vecR_{ij}}^3} \bc{ \vecd_i \cdot \vecd_j -3 (\vecd_i \cdot \hat{\vecR}_{ij})(\vecd_j \cdot \hat{\vecR}_{ij}) }
\end{align}
in a spherical tensor representation which will be useful throughout this thesis~\cite{Micheli2007,Gorshkov2011c}.
\Tref{ddint} is given in terms of the vector $\vecR_{ij} = \vecR_j - \vecR_i$ and its normalized form $\hat{\vecR}_{ij}$.
For electric (magnetic) dipoles, the constant prefactor is given by $\kappa = 1/4\pi\epsilon$ ($\kappa=\mu_0/4\pi$).
We focus on the distance-independent part $D_{ij} \equiv \Hdd \absv{\vecR_{ij}}^3/\kappa=\vecd_i \cdot \vecd_j -3 (\vecd_i \cdot \hat{\vecR}_{ij})(\vecd_j \cdot \hat{\vecR}_{ij})$ of the dipole-dipole interaction which can be written in terms of the spherical tensor $T^2(\vecd_i, \vecd_j)$ of rank two with components
\begin{align}
T^2_0(\vecd^i, \vecd^j) &= \frac{1}{\sqrt{6}} (d_i^+ d_j^- + 2d_i^0 d_j^0 + d_i^{-} d_j^{+}), \\
T^2_{\pm 1}(\vecd^i, \vecd^j) &= \frac{1}{\sqrt{2}} (d_i^{\pm} d_j^0 + d_i^0 d_j^{\pm}), \\
T^2_{\pm 2}(\vecd^i, \vecd^j) &= d_i^{\pm} d_j^{\pm}.
\end{align}
It is constructed from the two dipole moments which are themselves rank-one tensors with spherical components
\begin{align}
d^0_i = d^z_i, \qquad d_i^\pm = \mp\frac{1}{\sqrt{2}}\bb{d_i^x\pm i d_i^y}.
\end{align}
Using this, the dipole-dipole interaction can be written as a contraction of two rank-two tensors \cite{Brown2003}, leading to
\begin{align}
    D_{ij} &= -\sqrt{6} \: T^2(C)\cdot T^2(\vec{d_1},\vec{d_2})\\
           &= -\sqrt{6}  \sum_{m=-2}^{2} (-1)^m C^2_{-m}(\theta,\phi) \, T^2_m(\vecd_i,\vecd_j).
\end{align}
Here, $C^l_m(\theta,\phi)=\sqrt{\frac{4\pi}{2l+1}} Y^l_m(\theta,\phi)$ are the modified (unnormalized) spherical harmonics and $\theta, \phi$ are the spherical angles of the vector $\hat{\vecR}_{ij}$ in the coordinate system of the quantization axis. We can expand this expression to get
\begin{align}\tlabel{doperator}
    D_{ij} &= (1-3\cos^2 \theta) \bc{ d_i^0 d_j^0 + \frac{1}{2}\bb{d_i^+ d_j^- + d_i^- d_j^+} } \\
           &\quad -\frac{3}{\sqrt{2}}\sin\theta \,\cos\theta \bc{ (d_i^0 d_j^- + d_i^- d_j^0) \ef{+i\phi} - (d_i^0 d_j^+ + d_i^+ d_j^0) \ef{-i\phi} } \\
           &\quad -\frac{3}{2}\sin^2\theta \bc{ \ef{+2i\phi} d_i^- d_j^- + \ef{-2i\phi} d_i^+ d_j^+ }.
\end{align}
It is worth noting that the $T^2_{m=0}(\vecd_i, \vecd_j)$-terms in the first row conserve the ``internal'' angular momentum while the $m=1$ ($m=2$) terms in the second (third) row increase or decrease the internal angular momentum by one (two) quanta.

For most applications, we will be concerned with two-dimensional systems where the dipoles are aligned perpendicular to the plane. Then, the dipoles are also perpendicular to the interconnecting axis $\vecR_{ij}$, implying $\theta = \pi/2$. In this case, the $m=1$ terms drop out and the tensorial part reduces to
\begin{align}
    D^\text{(2D)}_{ij} = d_i^0 d_j^0 + \frac{1}{2}\bb{d_i^+ d_j^- + d_i^- d_j^+} -\frac{3}{2} \bb{ d_i^- d_j^- \ef{+2i\phi} + d_i^+ d_j^+ \ef{-2i\phi} }
\end{align}
For a one-dimensional geometry there are two high-symmetry geometries. If the dipoles are perpendicular to the lattice direction (say, the $x$-direction) we can set $\phi=0$, further simplifying $D_{ij}$ to
\begin{align}
    D_{ij}^{(1D,\perp)} &= d_i^0 d_j^0 + \frac{1}{2}\bb{d_i^+ d_j^- + d_i^- d_j^+} -\frac{3}{2} \bb{d_i^- d_j^- + d_i^+ d_j^+} \\
 &= d_i^z d_j^z + d_i^y d_j^y - 2 d_i^x d_j^x.
\end{align}
Conversely, if the dipoles point along the lattice direction, we can set $\theta=0$ in \tref{doperator} to get
\begin{align}
    D^{(1D,\parallel)}_{ij} &= -2 d_i^0 d_j^0 - \bb{d_i^+ d_j^- + d_i^- d_j^+} = -2 d_i^z d_j^z+ d_i^x d_j^x + d_i^y d_j^y
\end{align}

\subsection{Tilted electric field geometry}
\fig{.5}{lattice-geometry}{lattice-geometry}{Illustration of the relevant axes and angles. The lattice lies in the $xy$ plane while the electric field is tilted from the $z$ axis by an angle $\Theta_0$ and rotated around the $z$ axis by an angle $\Phi_0$ with respect to the $x$ axis. The direction of the vector $\vec{R}$, connecting two dipoles, is determined by the polar angle $\Phi$.}

\noindent
Here we consider a more general situation (see \tref{lattice-geometry}) where we have a two-dimensional system in the $xy$ plane and the DC electric field $\vec{E}$ points in an arbitrary direction~\cite{Gorshkov2011c}, determined by the spherical angles $\Theta_0, \Phi_0$:
\begin{align}
    \vec{\hat{E}}=\begin{pmatrix}
        \sin\Theta_0\cos\Phi_0 \\
        \sin\Theta_0\sin\Phi_0 \\
        \cos\Theta_0
    \end{pmatrix}.
\end{align}
As before, we are interested in the interaction between two dipoles which are separated
by the vector
\begin{align}
\vecR_{ij} = \begin{pmatrix}R_{ij} \cos\Phi_{ij} \\ R_{ij} \sin\Phi_{ij} \\ 0\end{pmatrix}.
\end{align}
For the angle $\theta$ between the dipole orientation $\vec{\hat{E}}$ and the interconnection line between the dipoles $\hat{\vecR}_{ij}$, we find the relation
\begin{align}
    \cos\theta = \vec{\hat{E}} \cdot \hat{\vecR}_{ij} &= \sin\Theta_0\bb{\cos\Phi_0\cos\Phi_{ij}+\sin\Phi_0\sin\Phi_{ij}}\\
                                                      &=\sin\Theta_0\cos(\Phi_{ij}-\Phi_0).
\end{align}
With the difference $\bar\Phi=\Phi_{ij}-\Phi_0$, we can express the relevant terms in the dipole-dipole interaction as
\begin{align} \tlabel{fmfunctions}
    f_0(\Theta_0, \bar \Phi)&\equiv 1-3\cos^2\theta = 1-3\sin^2\Theta_0\cos^2\bar\Phi, \\
    f_1(\Theta_0, \bar \Phi)&\equiv \sin\theta\cos\theta \ef{i\phi} = \sin\Theta_0 \cos\bar\Phi \bb{\cos\Theta_0 \cos\bar\Phi + i \sin \bar\Phi}, \\
    f_2(\Theta_0, \bar \Phi)&\equiv\sin^2\theta \ef{2i\phi} = \bb{\cos\Theta_0 \cos\bar\Phi + i \sin \bar\Phi}^2.
\end{align}
% which are easily seen to reduce to the old expressions in the case $\Theta_0=0$, implying $\theta=\pi/2$ and $\bar\Phi=\Phi_{ij}$.
In total, the tensorial part of the dipole-dipole interaction from \tref{doperator} is given by
\begin{align} \tlabel{tensortilted}
    D_{ij}(\Theta_0, \bar\Phi) = f_0(\Theta_0, \bar \Phi) &\big[ d_i^0 d_j^0 + \frac{1}{2}\bb{d_i^+ d_j^- + d_i^- d_j^+} \big] \\
      \qquad-\frac{3}{\sqrt{2}} &\Big[ f_1(\Theta_0, \bar \Phi)  (d_i^0 d_j^- + d_i^- d_j^0) - f_1(\Theta_0, -\bar \Phi) (d_i^0 d_j^+ + d_i^+ d_j^0) \Big]\\
      \qquad-\frac{3}{2} &\Big[ f_2(\Theta_0, \bar \Phi) d_i^- d_j^- + f_2(\Theta_0, -\bar \Phi) d_i^+ d_j^+\Big]
\end{align}
% The dipole-dipole interaction is symmetric under spatial inversion, implying:
% \begin{align}
%     f_m(\Theta_0, \bar \Phi + \pi) = f_m(\Theta_0, \bar \Phi).
% \end{align}
% Under complex conjugation, we find
% \begin{align}
%     f_m^*(\Theta_0, \bar \Phi) = f_m(\Theta_0, -\bar \Phi)
% \end{align}
% In addition, we can write
% \begin{align}
%     f_1(\Theta_0, \bar \Phi) = \frac{1}{2} \sin \Theta_0 \bc{\cos \Theta_0 (1+\cos 2\bar \Phi) + i \sin 2 \bar \Phi}
% \end{align}
% For small angles $\Theta_0$, we can rewrite
% \begin{align}
%     f_0(\Theta_0, \bar \Phi) &= 1 \\
%     f_1(\Theta_0, \bar \Phi) &= \frac{1}{2} \Theta_0 \bc{1+\ef{2 i \bar \Phi}}=\Theta_0 \cos \bar \Phi \ef{i \bar \Phi}\\
%     f_2(\Theta_0, \bar \Phi) &= \ef{2 i \bar \Phi}
% \end{align}
% Defining $\Theta_0^* = \pi/2 - \theta_\text{magic}=\text{asin}(1/\sqrt{3})$, we have
% \begin{align}
%     f_0(\Theta_0^*, \bar \Phi) = 1-\cos^2\bar \Phi
% \end{align}

\todo{Do we need the following paragraphs?}

\subsubsection{In-plane electric field}
For $\Theta_0=\pi/2$, the expressions above reduce to
\begin{align}
    1-3\cos^2\theta &= 1-3\cos^2\bar\Phi \\
    \sin\theta\cos\theta \ef{\pm i\phi} &= \pm \frac{i}{2} \sin 2\bar\Phi \\
    \sin^2\theta \ef{\pm 2i\phi} &= - \sin^2 \bar\Phi
\end{align}

\subsubsection{One-dimensional setup}
For a 1D setup we can set $\Phi_0=0$. The angle $\Phi$ is limited
to the values $0$ or $\pi$, indicated by $\sin\Phi=0$ and
$\cos\Phi=\pm 1=\text{sign}(X)$. Then, the expressions above reduce to
\begin{align}
    1-3\cos^2\theta &= 1-3\sin^2\Theta_0 \\
    \sin\theta\cos\theta \ef{\pm i\phi} &= \sin\Theta_0 \cos\Theta_0 \\
    \sin^2\theta \ef{\pm 2i\phi} &= \cos^2 \Theta_0
\end{align}

