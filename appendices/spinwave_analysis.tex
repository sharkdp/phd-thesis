\chapter{Spin-wave analysis}
\label{spinwave_analysis}

We present the derivation of the spin wave excitation spectrum
within the spin wave analysis. The basic approach is to start with the ground
states exhibiting perfect order, which are the correct ground states for the classical
model at the four points $\theta = 0,\pm \pi/2, \pi$.  Then, we introduce bosonic
creation and annihilation operators creating a spin excitation above the ground
state according to the Holstein-Primakoff transformation. The spin Hamiltonian
then reduces to a Bose-Hubbard model. In lowest order, we can ignore the
interactions between the bosonic particles, and obtain a quadratic Hamiltonian
in the bosonic operators, which is diagonalized using a
Bogoliubov-Valantin transformation.  The latter transformation deforms the
ground state and introduces fluctuations into the system.

In the following, we demonstrate the spin wave analysis for the most revealing case: the antiferromagnetic XY phase.
The generalization to the other ground states is straightforward.  Without loss of generality, we choose the anti-ferromagnetic order
to point along the  $x$ direction. The square lattice is bipartite, and we denote the two sublattices as A and B. Then, the anti-ferromagnetic
 mean-field ground-state is given by   $|G\rangle = \prod_{i\in A} \left|\leftarrow\right\rangle_{i} \prod_{j\in B} \left|\rightarrow\right\rangle_{j}$
 with spins on sublattice~A pointing in the negative $x$ direction, i.e., $S^x \left|\leftarrow\right\rangle = -\hbar/2 \left|\leftarrow\right\rangle$,
 and spins on sublattice~B pointing in the positive $x$ direction ($S^x \left|\rightarrow\right\rangle = \hbar/2 \left|\rightarrow\right\rangle$).
 Excitations on sublattice A are created by flipping a spin with the ladder operator $S^{x+} = S^z-i S^y$,
 while excitations on sublattice B are created via $S^{x-} = S^z+i S^y$.
 We apply a Holstein-Primakoff transformation to bosonic operators
\begin{align}
    S^z_i &= \frac{\hbar}{2} (a^{\phantom\dag}_i + a^{\dag}_i) \,\varphi(n_i),\\
    S^y_i &= \frac{\hbar}{2i} (a^{\phantom\dag}_i - a^{\dag}_i)\, e^{i {\vec K} {\vec R_i}} \,\varphi(n_i),
\end{align}
where the phase $e^{i {\vec K}{\vec R_i}}=e^{-i {\vec K}{\vec R_i}}$ accounts for the sublattice-dependent sign with ${\vec K} = (\pi/a, \pi/a)$. The factor $\varphi(n_i)=1-n_i$ is introduced to guarantee bosonic commutation relations for the operators $a_i$.  Here, we are interested in the leading order of the spin wave expansion, and can therefore set
 $\varphi(n_i) \approx 1$. The bosonic operators reduce to
%
\begin{align}
    \aop_i &= (S^z +i S^y e^{i {\vec K} {\vec R_i}} )/\hbar,\\
    \aopd_i &= (S^z -i S^y e^{i {\vec K} {\vec R_i}})/\hbar,
\end{align}
%
and the number operator $\nop_i = \aopd_i \aop_i = \frac{1}{2} + S_x e^{i {\vec K}{\vec R_i}}/\hbar$.


Expanding the spin Hamiltonian in terms of the bosonic operators leads to a Bose-Hubbard Hamiltonian for the spin wave excitations.
In leading order, we can neglect the interactions between the bosons and obtain the quadratic Hamiltonian
%
\begin{align} \tlabel{HamiltonianXYAF}
H/J = &\sin \theta \, \epsilon_{\vec K} \left(\frac{3 N}{4} - \frac{1}{2}\sum_i\left[\aopd_{i}\aop_i+ \aop_i \aopd_{i}\right]\right) \\
           &+  \frac{1}{4} \sum_{i\ne j} \frac{
    \chi_{i j} \bb{ \aopd_{i} \aop_{j}  + \aop_{i} \aopd_{j} }
  + \eta_{i j} \bb{ \aop_{i} \aop_{j}  + \aopd_{i} \aopd_{j} }
}{|{\vec R}_{ij}/a|^3}
\end{align}
with  ${\vec R}_{ij}={\vec R}_i - {\vec R}_j$,  $N$ the number of lattice sites, and the coupling the terms
$\chi_{i j}=\cos\theta+\sin\theta e^{i {\vec K} {\vec R}_{ij}}$ and $\eta_{i j}=\cos\theta-\sin\theta e^{i {\vec K} {\vec R}_{ij}}$ including the
 anti-ferromagnetic ordering.
Introducing the  Fourier representation $\aop_i =  \sum_{\vec q} \aop_{\vec q}e^{-i {\vec q} {\vec R}_i}/\sqrt{N}$,
the terms involving the bosonic operators in \tref{HamiltonianXYAF} reduce to
%
\begin{align}
\frac{1}{4} \sum_{\vec q} \bigg[
\left( \cos\theta \, \epsilon_{\vec q} +\sin\theta \, \epsilon_{\vec q+K} - 2 \sin \theta \epsilon_{\vec K}\right)\Big(\aopd_{\vec q} \aop_{\vec q}  + \aop_{\vec q} \aopd_{\vec q}\Big) \\
+\left( \cos\theta \, \epsilon_{\vec q}-\sin\theta \, \epsilon_{\vec q+K} \right)\Big(a^{\phantom\dag}_{\vec q} a^{\phantom\dag}_{-\vec q}
+ a^{\dag}_{\vec q} a^{\dag}_{-{\vec q}} \Big) \bigg].
\end{align}
%
The diagonalization of this Hamiltonian is straightforward using a standard Bogoliubov transformation
with $\bopd_{\vec q} = u_{\vec q}\, \aopd_{\vec q} - v_{\vec q}\, \aop_{-\vec q}$.
Then, the Hamiltonian takes the form
%
\begin{align}
H = \frac{3J N \sin \theta\, \epsilon_{\vec K}}{4}  + \sum_{\vec q} E^{\mbox{\tiny XY-AF}}_{\vec q} \bb{\bopd_{\vec q}\bop_{\vec q} + \frac{1}{2}}
\end{align}
%
with the spin-wave excitation spectrum $E^{\mbox{\tiny XY-AF}}_{\vec q}$. In addition, the coefficients  for the Bogoliubov transformation
are given by
%
\begin{align}
u_{\vec q} , v_{\vec q} = \pm\sqrt{\frac{1}{2} \left( \frac{\cos\theta \, \epsilon_{\vec q} +\sin\theta \, (\epsilon_{\vec q+K} - 2 \epsilon_{\vec K})}{2\, \mathcal{E}_{\vec q}}\pm 1\right)},
\end{align}
%
with $\mathcal{E}_{\vec q} \equiv E^{\mbox{\tiny XY-AF}}_{\vec q} / J $.
The property $u^2_{\vec q}-v^2_{\vec q}=1$ asserts that the transformation is canonical.
In addition, the ground state obeys the condition $b_{\vec q} \left|\text{vac}\right\rangle = 0$, and the ground state energy per spin at zero temperature $T=0$
reduces to $e_{\mbox{\tiny XY-AF}}$, see \cref{ab:table1}.

We are now able to check the validity of the spin wave approach self-consistently: the deformation of the ground state by the spin wave analysis
provides a suppression of the anti-ferromagnetic order $ m \equiv  \Delta m - \frac{1}{2}=\left\langle S^x_i e^{i {\vec K} {\vec R_i}} \right\rangle / \hbar$,
and thus
%
\begin{align}
    \Delta m= \int \frac{d{\vec q}}{v_0} \, \meanv{ \aopd_{\vec q} \aop_{\vec q} } = \integralf{\vecq}{v_0}  \bc{ v_{\vec q}^2 + (2 v_{\vec q}^2 + 1) f_{\vec q} },
\end{align}
%
where $f_{\vec q} = \meanv{ \bopd_\vecq \bopd_\vecq } = \left[\exp(E^{\mbox{\tiny XY-AF}}_{\vec q}/ T)-1\right]^{-1}$ accounts for the thermal occupation of the spin waves. At zero temperature $T=0$, this expression converges and  we obtain $\Delta m \approx 0.03$ for $\theta = \tilde\theta_c$ as well as $\Delta m = 0.39$ for $\theta \approx \frac{\pi}{4}$. In turn, at finite temperatures $T>0$, the low momentum behavior of  the integrand scales as $|{\vec q}|^{-2}$, and therefore $\Delta m$ diverges logarithmically: the long range order is destroyed by the thermal spin wave fluctuations, and gives rise to the well known
quasi long-range order in analogy to short range XY models.

Finally, the spin wave analysis also allows us to analyze the correlation functions
 $c_{\alpha\alpha}({\vec R}_{ij})=\langle S^\alpha_i S^\alpha_{j} e^{i {\vec K} {\vec R}_{i j}} \rangle$. Using
 the translational invariance of our system, the correlation functions reduce to
%
\begin{align}
c_{\alpha\alpha}({\vec r}) = \int \frac{d{\vec q}}{v_0} \, c_{\alpha\alpha}({\vec q+K}) \, e^{-i {\vec q} {\vec r}}
\end{align}
%
with $c_{\alpha\alpha}({\vec q}) = \langle S^\alpha_{\vec q} S^\alpha_{-\vec{q}}\rangle$.
