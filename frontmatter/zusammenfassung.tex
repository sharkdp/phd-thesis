\chapter*{Zusammenfassung}

\begin{otherlanguage}{ngerman}

Die vorliegende Arbeit beschäftigt sich mit der Untersuchung verschiedener
Modellsysteme im Rahmen der ultrakalten Quantengase.
Im Mittelpunkt stehen dabei neuartige Verfahren um Quantenzustände mit topologisch nichttrivialen Eigenschaften mittels dipolaren Wechselwirkungen zu realisieren.
Ein prominentes Beispiel für einen Zustand mit topologischen Eigenschaften zeigt sich im Quanten-Hall Effekt.
Die exakte Quantisierung der Hall-Leitfähigkeit kann durch das Auftreten einer topologischen Invarianten verstanden werden.
Die Robustheit entsprechender physikalischer Effekte gegenüber äußeren Störungen macht topologische Materialien dabei interessant für Anwendungen.
Entdeckt wurde der Quanten-Hall Effekt in zweidimensionalen Elektronengasen bei extrem tiefen Temperaturen und hohen Magnetfeldern.
Die schwierigen experimentellen Bedingungen, sowie eine Reihe offener Fragen, besonders im Bereich des fraktionalen Quanten-Hall Effekts, motivieren daher die Frage nach alternativen Systemen.

Seit einigen Jahren sind Experimente auf dem Gebiet der ultrakalten Quantengase so weit fortgeschritten, dass routinemäßig neuartige Modellsysteme simuliert werden können.
Ein Abschnitt dieser Arbeit beschäftigt sich daher mit der Realisierung des Quanten-Hall Effekts in ultrakalten Gasen.
Ein Problem besteht darin, den Effekt des Magnetfelds auf Elektronen mit elektrisch neutralen Atomen zu simulieren.
Eine mögliche Lösung bedient sich einer exakten Analogie zwischen geladenen Teilchen im Magnetfeld und neutralen Teilchen in einem rotierenden System, wobei die Rotationsfrequenz die Rolle des Magnetfeldes übernimmt.
Die Corioliskraft im rotierenden System verhält sich dabei beispielsweise wie die Lorentzkraft im Magnetfeld.
Um das zweidimensionale Quantengas in Rotation zu versetzten, verwenden wir den Relaxierungsmechanismus in dipolaren Systemen.
Dabei wird interner Drehimpuls der Atome durch die Dipol-Dipol Wechselwirkung in eine externe Rotation umgewandelt.
Der Vorteil dieser Methode besteht darin, dass nicht die Drehfrequenz des Systems gesteuert wird, sondern direkt der Gesamtdrehimpuls.
Hierdurch kann ein experimentelles Problem umgangen werden, das auftritt, wenn die Rotationsfrequenz vergleichbar wird mit der Fallenfrequenz.

Der zweite große Teil dieser Arbeit beschäftigt sich auf eine andere Weise mit topologischen Quantenzuständen.
Wie Haldane 1988 zeigte, lässt sich die quantisierte Hall-Leitfähigkeit und die damit verbundene topologische Eigenschaft auch ohne die Präsenz eines Magnetfeldes realisieren.
Er betrachtete ein Modell auf dem hexagonalen Gitter, wobei die Zeitumkehrinvarianz durch das Auftreten von komplexen Tunnelraten gebrochen wird.

In unserem Modell gehen wir von dipolaren Teilchen, beispielsweise polaren Molekülen oder Rydberg Atomen, aus, die in einem zweidimensionalen Gitter fixiert sind.
Wir sind an der Dynamik der internen Anregungen dieser Dipole interessiert, die durch die Dipol-Dipol Wechselwirkung getrieben wird.
Insbesondere können diese Anregungen zwischen verschiedenen Teilchen ausgetauscht werden, und sich damit ähnlich wie tunnelnde Elektronen verhalten.
Betrachtet man zwei verschiedene Anregungen mit unterschiedlichem internen Drehimpuls, dann können diese ineinander umgewandelt werden.
Dabei tritt ein komplexer Faktor auf, der durch die Gesamtdrehimpuls-Erhaltung der Dipol-Dipol Wechselwirkung kommt.



\end{otherlanguage}
